\thesischapterexordium

\section{研究工作的背景与意义}

随着数字摄像机等新媒体技术的广泛运用,人们的日常生活对于图片和视频的需求快速增长,因此这一方面数据量很大。尤其是近几年深度学习技术被用于计算机视觉领域的很多子方向,包括了人脸检测\citing{8272675},人脸识别、目标检测\citing{7780460}、语义分割\citing{10.1007/978-3-319-24574-4_28}、人体姿态估计与识别等。其中,人体姿态估计是最热门的一个方向,是当前研究的一个热点方向。人体姿态估计是一种定位人体的肩膀,手肘和脚踝等人体关键位置的技术,而三维人体姿态估计试图恢复人体的三维立体姿态,为进一步的视频内容理解等任务打下基础,有很多的应用场景,具体体现如下:

(1)视频监控

\citing{xiongzihua}传统的监控系统还是要依靠人眼去判断监控器中的异常行为,在一些角度受限和光线条件不好的情况下,安保人员对于视频内容的理解与分析存在很大的局限性,而且只是具有数据收集,数据存储的功能,有低效率和滞后性的特点。因此需要提高其自主分析的能力,利用更加丰富的,有效的信息辅助决策,新一代监控系统,在面对老人跌倒,工人错误等异常情况下,能够代替工作人员分析并对应地发出警报。

(2)人机交互

在现代社会,人们不能满足于键盘,触摸屏等传统的交互方式,因为人类的肢体语言包含着更丰富的信息,所以人体姿态估计这一项研究逐渐得到青睐。人机交互使得机器能够识别人体姿态,可以在自助服务,无人零售等新应用场景下发挥出更大的优势,这使得能够为将来人与机器人的交流带来更多的可能。

(3)康复训练

凭着比人工医护费用低的优势,让机器人参与患者的康复训练,如何帮助患者恢复四肢运动能力的这一项技术,逐渐成为现代医疗的核心。在不借助传感器的情况下,使得患者的运动情况得以和计算机交互,使得医护人员可以清楚地了解到患者的肢体恢复状态,从而给出更棒的康复建议,从而在真正意义上地帮助患者的恢复。

(4)新媒体娱乐

这几年新媒体技术发展迅速,手机成为人们日常生活的重要组成部分,除了看视频和阅读文字的业务以外,AR、VR等娱乐方式逐渐成为热门,例如前段时间字节跳动公司开发的在线尬舞机业务,就是成为了一款流行产品。其使用人体姿态估计技术,以舞蹈姿势的提示形式,帮助用户完成指定舞蹈动作。

\section{人体姿态估计的国内外研究历史与现状}

对于摄像原理而言,因为最后形成的图像,是将三维信息投射到二维平面的处理结果,这不断转换过程中,可能会丢失部分重要信息,导致人体姿态估计存在很多歧义性的理解。而且还受到拍摄角度的限制和复杂外部环境的影响,存在自遮挡和被遮挡等诸多问题,导致传统人体姿态估计算法的效果很差。

传统的人体姿态估计的算法主要是基于图结构(Pictorial Structures,PS)\cite{PS}模型,就是建立人体关节点和躯干运动之间的对应逻辑关系。但是近几年,随着深度学习的新算法在计算机视觉领域的广泛应用,有些科研人员致力于研究复杂环境下的人体姿态估计,标注了很多大规模的图像数据集,以方便学术研究和商业应用,其中的代表有MPII,COCO等人体姿态估计数据集。

目前学术界主流采用的是卷积神经网络的结构,其特点是通过大量的样本学习特定图像特征,实现特征学习并且完成对应的应用目标。Toshev\citing{517044}等人在2014年提出的DeepPose模型,卷积神经网络第一次被用于人体姿态估计中,通过多阶段逐步回归人体关键点位置,开辟了人体姿态估计的新纪元。因为回归坐标法的可拓展性较差,然而基于热力图的人体姿态估计判断准确度极高,因此逐渐成为主流思路,其原理是将热力图中概率响应最大的位置作为人体关键点位置。

而对于多人姿态估计的研究,主要分为自上而下(Top-Down)和自下而上(Bottom-Up)两种策略方案。

自上而下的策略是目标检测技术与单人姿态估计的结合,有以下典型算法:G-RMI (Google, 2017)\citing{wild}使用Faster-RCNN作为人体检测器,姿态部分使用ResNet作为主干网络来估计热力图和偏差矩阵,因为得到热力图之后往往还需要一个归一化的操作才能得到关节点的坐标,而这个过程中由于网络的下采样过程,热力图势必分辨率比原图更小,所以得到的坐标会出现偏移,因此又估计了一个偏差矩阵来补偿掉这种量化误差。CPN\citing{CPN}采用的网络结构是一个U-Shape的结构,多尺度特征信息的融合是网络设计的一个很大的目标,作者使用 GlobalNet (global pyramid network)来处理关键点,GlobalNet中包含了下采样和上采样(插值非转置卷积)的过程。\citing{8953968}这项工作主要是对网络结构的改进,主要创新点在于加入channel shuffle和注意力机制。MSRA Bin Xiao等人的工作
字节跳动公司\citing{simple_baseline}使用Deconvolution来做上采样,网络中也没有不同特征层之间的跨层连接,和经典的网络结构Hourglass和CPN相比都十分简洁。Kocabas\citing{MultiPoseNet}使用了两个子网络,一个用来输出关键点和序列的热力图,另一个是监测器器,用来输出人体的监测框,然后将这两种输出送到Pose Residual Network中,得到最终的姿态。上海交通大学卢策老师\citing{Li2018CrowdPose}这项工作主要是要处理拥挤场景下的多人姿态估计问题,并且在MPII, COCO和AI Challenger数据集基础上做了一个新的拥挤标签。连接关节点是通过关节点间的距离来建立,个人点通过检测的人数来建立,两者之间的边通过看是否有贡献来建立。由此建立了一个人-关节的图,也就转化到了图论问题上,目标就是最大化二分图中的边权重。使用updated Kuhn-Munkres解决这个问题。

而自下而上的策略并不依赖于人体检测器,关键在于如何对图像中所有人体关键点分类到对应的每个人体,有以下典型算法:OpenPose\citing{8099626}是目前Bottom-up方法中影响最大的工作,网络结构基于CPM改进,网络包含两个分支,一个分支预测热力图,另一个分支预测亲合力场(part affine field),转化为二分图匹配(bipartite graph)的问题,使用匈牙利算法求解。DeeperCut\citing{9010416}使用深层的残差网络架构来检测身体部分,使用图像条件成对项来做优化,可以将众多候选节点减少,通过候选节点之间的距离来判断该节点是否重要。

\section{本文的主要贡献与创新}

本论文基于沙漏网络模块为骨干网络,结合人体骨骼模型的几何特点以及成像原理,设计了深度回归模块,如何提高人体姿态估计的准确度作为重点研究内容,主要创新点与贡献如下:

基于人体骨骼模型,采用树形模型,采用骨盆作为人体中间点,对图像中的人体信息进行语义分割,提取出单人信息,实现多人姿态估计的任务。

基于沙漏模块的基础上,设计了八阶堆叠沙漏网络作为训练网络的骨干网络,使得平均关节误差MPJPE优于普通沙漏网络。

基于摄像成像原理,结合单孔成像,对人体姿态的第三维坐标进行了几何约束,使得第三维坐标的精确性得到了提高。

采用两阶段的算法思路,整体算法框架运行顺利,使得平均关节误差MPJPE达到56mm,满足任务书中的62mm的精度要求。

\section{本论文的结构安排}

本文的章节结构安排如下:

第一章,分析了人体姿态估计的研究起源,并详细介绍了人体姿态估计未来的四大应用场景,分析并研究了人体姿态估计的国内外现状,并阐述了全文研究内容和章节的段落安排。

第二章,介绍了深度学习基本概念,包括卷积神经网络,以及输入通道和输出通道的定义。然后介绍了残差模块,沙漏模块,堆叠沙漏网络的设计原理。

第三章,介绍了人体姿态估计的表示方法,以及衡量人体姿态估计算法优劣的误差度量指标,以及目前学术界主流数据照片集的标注方法。并且介绍了摄像单元的成像原理,并结合单孔成像原理,设计了一个深度回归模块,用于提升整个算法框架。

第四章,着重在MPII,COCO数据集上做了实验测试,并且在比较各阶堆叠沙漏网络的识别效率后,得出了运算效率的最佳堆叠沙漏网络的结构。然后比较了下不同数据集的标注方法对识别效率的影响。最后,对深度回归模块的优化效果进行了验证,确实提高了识别精度。分析了整体算法的优劣,并与其他同类算法进行对比。