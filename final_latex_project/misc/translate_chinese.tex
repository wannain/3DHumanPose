
\thesistranslationchinese
\section{堆叠沙漏网络架构}

\subsection{沙漏设计}

沙漏的设计是由需要捕捉每一个尺度上的信息驱动的。虽然局部证据对于识别面部和手部等特征至关重要,但最终的姿势估计需要对全身有连贯的理解。人的方位、四肢的排列以及相邻关节之间的关系是在图像中不同尺度下最容易识别的线索之一。沙漏是一个简单,最小的设计,有能力捕捉所有这些特征,并将它们结合起来,输出像素级的预测。 

网络必须具有某种机制,以便跨规模有效地处理和整合功能。一些方法通过使用单独的管道来解决这个问题,这些管道在多个分辨率下独立处理图像,然后在网络中组合特征[15,18]。相反,我们选择使用带有跳过层的单个管道来保留每个分辨率的空间信息。该网络以 4x4 像素的分辨率达到最低,允许应用更小的空间过滤器来比较图像整个空间的特征。 

沙漏的设置如下:卷积层和最大池层用于将特征处理到非常低的分辨率。在每一个最大池步骤,网络分支并在原始预池分辨率下应用更多的卷积。在达到最低分辨率后,网络开始自上而下的跨尺度上采样和特征组合序列。为了将两个相邻分辨率的信息汇集在一起,我们遵循汤普森等人描述的过程。[15]并对较低分辨率进行近邻上采样,然后对两组特征进行元素相加。沙漏的拓扑结构是对称的,所以在向下的每一层都有一个对应的层向上。 

在达到网络的输出分辨率后,应用连续两轮1x1卷积来产生最终的网络预测。网络的输出是一组热图,其中对于给定的热图,网络预测每个像素处存在关节的概率。完整的模块(不包括最后的 1x1 层)如图 3 所示。

\subsection{分层实施}

在保持整体沙漏形状的同时,层的具体实现仍有一定的灵活性。不同的选择会对网络的最终性能和培训产生适度的影响。我们探讨了在我们的网络层设计的几种选择。最近的工作已经显示了使用1x1卷积的简化步骤的价值,以及使用连续的较小滤波器捕获较大空间上下文的好处。[12,14]例如,可以用两个单独的3x3过滤器替换5x5过滤器。我们测试了我们的整体网络设计,根据这些见解在不同的层模块中交换。在从带有大滤波器的标准卷积层切换到像He等人提出的剩余学习模块这样的新方法之后,我们经历了网络性能的提高。[14] 以及基于“盗梦空间”的设计[12]。在使用这些类型的设计改进了最初的性能之后,对层进行的各种额外的探索和修改对进一步提高性能或训练时间几乎没有什么帮助。 

左图:我们在整个网络中使用的剩余模块[14]。右图:中间监督过程示意图。网络分裂并产生一组热图(蓝色轮廓),其中可以应用损耗。1x1卷积重新映射热图,以匹配中间特征的通道数。这些是与前面沙漏的特征一起添加的。 

我们的最终设计充分利用了残差模块。从不使用大于3x3的筛选器,并且瓶颈限制了每一层的参数总数,从而减少了总内存使用量。我们网络中使用的模块如图4所示。为了将其放到整个网络设计的上下文中,图 3 中的每个框表示一个单独的残差模块。 

以256x256 的完全输入分辨率运行需要大量的GPU内存,因此沙漏的最高分辨率(因此最终输出分辨率)为 64x64。这不会影响网络产生精确联合预测的能力。整个网络从一个7x7 卷积层开始,步长为 2,接着是一个剩余模块和一轮最大池,将分辨率从 256 降到 64。图 3 所示的沙漏之前有两个后续的残差模块。在整个沙漏中,所有残差模块输出 256 个特征。 


