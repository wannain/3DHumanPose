	
\begin{chineseabstract}

随着机器学习技术的快速发展,以大数据训练方式进行的目标检测研究,得到了快速的发展。而人体姿态估计就是其中一个热门,这是因为人体姿态具有丰富的信息,可以提取出很多有用的特征信息。目前主要采用骨干网络训练人体关键点,用热力图来判断准确的关节点位置信息。

本文是基于三维人体姿态估计的研究背景,该算法框架主要分为两大步骤,其一就是利用卷积神经网络结构对图像的二维姿态进行分析判断,得到对应的二维信息,其二就是利用图像的特征信息来估计人体在空间中的深度信息,最终实现三维人体姿态估计。

堆叠沙漏网络是基于残差模块的,在训练过程中可以有效地避免,反向传播过程中梯度消失的问题。随着大量数据的训练,在MPII数据集上,实验证明更高阶的堆叠沙漏网络,可以有效提高识别人体姿态估计的准确度和平均关节误差,最终以一个很高的精度完成二维人体姿态估计。

实际应用场景中,深度信息的重建一直是三维人体姿态估计的重要研究问题,在上述八阶堆叠沙漏网路的算法框架基础上,设计了人体的几何约束条件,提出了一种深度回归模块。该模块旨在实现,利用二维信息估计三维姿态的功能,以及如何将人体关键点与其人体中心之间的联系。实验结果证明,该模块使得整体算法在关键点识别误差上有了很大的提高,并且实现了多人姿态估计。

\chinesekeyword{人体姿态估计,堆叠沙漏网络,成像原理,深度回归模块}
\end{chineseabstract}

